\section{Summary}
In summary, three experiments, or portions thereof, were modified using micro-controller technology coupled with either Python programming or provisions for reading data via smart-phones.
In the Gas Detector and Statistics experiment, an Arduino compatible PIN GM was used in conjunction with an Arduino Nano and Raspberry Pi.
Students were expected to write their own Arduino data acquisition code and Python data storage code.
Students, overall, generally enjoyed this experiment and felt that the inclusion of the new technology helped to make the experiment more interesting.
Students particularly felt that they got extra experience, and one could infer, increased self sufficiency working with this experiment.
Students felt that there was more transparency in how the data was collected, stored, and analyzed compared to a standard experiment that operates as more of a black box.
In short, there are only minor improvements that we would make to this lab.
Namely, in finding better ways to present material in the lab manual.
Another improvement would be to, possibly, allow students to start writing code prior to the experiment to allow sufficient time for debugging.

In the Multiple Scattering portion of the Detector Principles experiment, a single, more traditional, gas GM was replaced by a series of four Arduino compatible gas GMs controlled by an Arduino Uno with data stored on a Raspberry Pi.
In this experiment, the goal of modernization was to simply reduce the data collection time by increasing the number of data points collected simultaneously using modern micro-controller technology.
Another goal was to find a better way to store the data using the Raspberry Pi.
Students had a little less freedom to develop their own programs in this experiment compared to that of the Gas Detector and Statistics experiment.
This was intentional as the Multiple Scattering experiment comprises one third of the amount of experiments in the Detector Principles lab, which is already typically running short on allotted time.
Students, overall, appreciated the efforts of modernizing this lab.
They liked that there were new technology that they had the opportunity to try out.
The students responded that they would like to have further automatization of the data collection as well as controlling the angular position of the detectors using the Arduino.
Controlling the angle would be straight-forward using a stepper motor, or something similar.
The automation of the data collection could be done using Python scripting, or something similar by the students.
In short, there are some minor improvements that can be done to improve the experiment for the students.

In the Stern-Gerlach, the third and final experiment which we added Arduino technology to, we simply replaced all of the analog readouts with Bluetooth serial readouts.
Thus, students would be able to use their own smart-phones with Android serial terminal applications to read the data.
Several issues were found with this mode of data acquisition.
The most important of these being that the Apple iStore does not support a free application that is compatible with the experiment for students with Apple devices.
Secondarily, some students do not have the ability to keep their devices charged.
Thirdly, students could only listen to a single Bluetooth+Arduino combination.
Lastly, there was not a good way of storing the data in a way that students could utilize it.

After several attempts using the smart-phone readout, it was abandoned in favor of USB to laptop readout.
This worked better, solving the issue with the iOS app and storing data.
However, some students still had difficulty because they often had an insufficient number of USB ports for the experiment.
In addition, even if the students had sufficient USB ports to accommodate the large number of Arduinos used in the experiment, they were still unable to read data from more than one Arduino at the same time.

Students, overall, generally liked the type of technology that was being introduced into this experiment.
However, students would have appreciated the experiment changes that were made if they worked with little to no difficulty.
Additionally, students voiced that they would like more freedom to make changes to the coding and data acquisition process.
Most of this can be accomplished easily by introducing a Raspberry Pi and reducing the number of Arduinos needed to perform all of the measurements required for this laboratory.

All of these changes that we made to the laboratory experiments were made in an attempt to modernize them and engage students in a more positive way.
Based on the feedback from the students, we would say that we were, overall, successful in doing so.
We also understand that there are certain cases where some, not insignificant, improvement is necessary prior to the next iteration of the laboratory course.
However, thanks to the student feedback we know where most of the important focus points for the discussed experiments are.
Also, thanks to the overwhelmingly positive feedback, we also know that students enjoy working with this type of equipment and enjoy being given more freedom to develop theor own solutions to collecting and analyzing the data in most experiments.
Thus, this gives a starting point for not only improving other existing experiments within the course, but also in designing new experiments moving forward.
