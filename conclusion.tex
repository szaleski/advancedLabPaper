\section{Student Response to the New Hardware and Experimental Procedure}
After making these changes, we found the following from the students...
The changes to the advanced lab experiments outlined in the previous three sections were implemented in the previous Winter 2019 semester.
At the end of each of the experiments the students were given a short questionnaire related to the changes made to the specific experiment that they performed.
The focus of the questionnaire was to understand if there were any perceived improvements from the students' perspective or if they benefitted in some way from the changes.
The questionnaire included questions related to the students' previous experience working with the microcontroller hardware and programming.
It also included questions to guage if the students enjoyed, or felt some benefit from performing the experiment.
The last portion of the exam was free response and asked the students to identify what they enjoyed about the experiment as well as what they thought could be further improved.

The questionnaires were collected and recorded promptly after they were answered by the students.
This was done in order to adjust the lab experiment quickly and efficiently if there were issues using the equipment, or if a portion of the procedure was not working well.
We present and discuss the results of the student response to these changes to the lab experiments.
Where appropriate, we also discuss any changes that were made in the middle of the semester for these experiments.

\subsection{Gas Detectors and Statistics}
The first two items in this questionnaire asked about the students' prior experience with Arduino+Raspberry Pi microcontrollers and Python programming.
These questions were only intended to get a coarse guage of the students' command and level of experience with these items.
12 students responded that they had no prior experience with Arduino and/or Raspberry Pi, while three responded that they had 'some' prior experience with these microcontrollers.
Here, some is defined to be less than three years of experience.
There were two other oprtions that students could have selected: 'significant' or 'expert'.
Significant was defined to be more than 3 years of experience, while expert would be considered as a fluent user of these items.
As a reference, 16 students responded to the questionnaire, however, not all students responded to all questions.
This suggests that the students that participated in this experiment did not have, collectively, much experience working with these microcontrollers prior to taking the course.

Students had much more experience using Python programming language.
Only a single student responded that they had no prior experience with Python.
13 students responded that they had 'some' experience.
One student answered that they had 'significant' experience.
No students considered themselves experts at writing Python code.
The degrees of this question were the same as the previous.
The student repsonses to these questions are shown in Tab. 1.



Questions three and four asked if the students enjoyed using the Raspberry Pi and the Arduino, respectively.
The goal of these questions was to understand if the students enjoyed using the microcontrollers and, from this, infer if future students might enjoy working with them as well.
The results for these questions are displayed in Tab. 2.
9 students agreed, or strongly agreed, that they enjoyed using the Raspberry Pi while 5 responded neutrally.
12 students agreed, or strongly agreed, that they enjoyed using the Arduino, while only a single student disliked working with it.
Overall, this would suggest that the students that performed this experiment enjoyed working with the microcontrollers and would infer that future students are likely to enjoy using them as well.

Question five tried to determine the students felt that working with the combination of Raspberry Pi, Arduino, and Python programming made the experiment more interesting when compared to other experiments within the lab course.
These results are shown in Tab. 3.
13 students agreed, or strongly agreed, that the combination of new technologies made for a more interesting experiment to work with.
Only one student remained neutral on this response.

Questions six, seven, and eight asked the students to evaluate, qualitatively, how much they felt that they learned about working with the Raspberry Pi, Arduino, and Python, respectively.
The responses to each of these questions can be found in Tab. 4.
Eight students reported that they gained some to alot knowledge about working with the Raspberry Pi.
Four student felt that they learned very little, or nothing about working with the Raspberry Pi.
Two students responded with a 'not applicable' answer.
12 students learned some to alot of new information about working with the Arduino, while two students felt that they learned very little, or nothing.
In response to the item on Python, 4 students responded that they learned some to alot of new items realted to programming.
Nine students responded that they learned nothing, or very little about Python.
One student responded with 'not applicable'.
As expected from the earlier responses, students felt that they learned more about the microcontrollers than they did from working with Python.


Students were given a chance to elaborate more on the experiment with free responses to what they liked about working with the new technology, what they didn't like about it, and any suggestions for improvement moving forward.
The free responses to what they enjoyed about the experiment fit into seven different categories.
There were a total of 15 responses to this portion.
The first category to be discussed is that of students enjoying solving a complex problem (at least partially) on their own.
Two students responded in a vein similar to ``I enjoyed writing my own computer program to perform this experiment''.
These students appreciated being able to write their own Python and/or Arduino program to collect and store data to solve an experimental problem that they were given.
A related category is using 'old' knowledge to solve a new problem.
Here old knowledge simply refers to that which the student already had prior to the start of the experiment.
This student responded that applying their programming knowledge to solve a physics problem was what they liked about the experiment.

Three students repsonded that they liked the choice of programming language(s) which they had to utilize to perform the experiment.
These students either liked that they enjoyed working with a language that they were already familiar with, learning a new language, or even working with two different languages within the same experimental framework.
Four students reported that they simply enjoyed learning very generally about working with the Raspberry Pi or Arduino microcontrollers.
One student stated that they liked that the microcontrollers were small and could be used for many different applications.

Two students liked that the new microcontroller technology was easy to use.
Two students also found that using the new technology made understanding how data is measured and recorded more apparent.

The students were also asked about what they didn't like about the experiment.
There were 15 responses to this item that fit into four different general categories.

The first category, receiving by far the largest number of complaints, is related to experiment logistics.
Eight students complained about either not having enough space for the experiment (4), having to switch the monitor connection between the Windows PC and the Raspberry Pi (3), or having a US keyboard layout rather than German (1).
These complaints were addressed within the first few lab sessions to provide more space, an additional monitor so that each computer has it's own, and to provide a German keyboard if they would want to use it.

The second category is related to computer program debugging.
Four students report that they were unable to debug their Arduino or Python program within the alloted time of the experiment without help from the experiment supervisor.
It was difficult to directly address this particular item in the middle of the semester.
Moving forward, it would be best that the instructions recommend that the students attempt to write the programs prior to starting the experiment or to supply the students with a known working software.
Though, the latter would defeat the purpose of allowing them to gain the experience of writing their own code.

Two students suggested that the experiment manual could have better descriptions of the some of the Python libraries and data analysis.
This material will be improved in the second iteration of the lab manual.
And one student complained that the Raspberry Pi was not necessary for the experiment.
The authors' response to this particular item is that it is true that one can collect the data with the GM counter, and Arduino only, using an Arduino compatible data-logging shield.
However, one of the intentions of the upgrade experiment is to expose the students to multiple microcontroller technologies.
Many experimenters, scientists, and electronics enthusiasts use the Raspberry Pi as a single-board PC.
It was simply envisioned to be used as an inexpensive computer to store and analyze data.
We respect the student's astute observation, but note that it misses one of the learning target items.

In the final item that the students were asked to respond to, students were allowed to propose suggestions to the experiment for the next iteration.
All of the responses repeat those from the previous two items. Thus we will not elaborate further on them.

We learned some important lessons to guide us in developing microcontroller improvements from this particular questionnaire.
An overwhelming percentage of students came in with little, or no, experience working with microcontrollers.
However, the same students enter the lab with sufficient programming experience that they can, with a guided manual, develop their own data storage code using a programming language that they are already comfortable with.
They can also develop a simple working data acquisition program using a program that they have not used before.
The students responded that they enjoy this as well as see it as an improved way for them to collect data.
From this we can say, at least for this small sample size, that this new microcontroller technology presented in a user-friendly way is an improved way to engage students in a university laboratory setting.

\subsection{Detector Principles}
Owing to the similiarity of hardware between the Gas Detector and Statistics lab and this one, many of the questions asked to the students are similar.
No Python programming was expected, thus some of the Python questions that occur in the previous discussion are substituted for similar ones regarding the Arduino compatible GM counter.
For reference 17 students responded to the questionnaire, but not all students responded to all of the questions.

In the first question, we ask the students how much prior combined experience they had working with Arduino and Raspberry Pi microcontrollers.
11 students responded that they had no experience with Arduino or Raspberry Pi before starting the experiment.
Six student answered that they had 'some' prior experience with these microcontrollers.
No students responded that they had 'significant' or 'expert' experience.
These responses have the same designation as the first two questions in the questionnaire related to the Gas Detector and Statistics experiment.
These responses are shown in Tab. 5.

Questions two and three are the same as three and four from the Gas Detector and Statistics questionnaire.
13 students agreed, or strongly agreed, that they enjoyed working with the Raspberry Pi while two students were neutral and one responded with 'not applicable'.
13 students agreed, or strongly agreed, to enjoying using the Arduino.
One student was neutral and three responded with 'not applicable.
These results are shown in Tab. 6.

Question four asks the students to respond to how interesting they found the new technology.
These results can be found in Tab. 7.
11 students agreed, or strongly agreed, that the new combination of the Raspberry Pi, Arduino, and GM counter made the experiment more interesting.
Two students disagreed, while three students responded neutrally, and one responded with 'not applicable'.

Questions five and six ask the students to evaluate how much the felt that they learned by working with the Raspberry Pi and Arduino, respectively.
