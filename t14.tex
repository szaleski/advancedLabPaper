\section{Stern-Gerlach Experiment}
An update to the readout of the Stern-Gerlach Experiment...
The final experiment that was updated with microcontroller technollogy was the Stern-Gerlach experiment.
In this experiment, students investigate the diffraction effects exhibited on particles passing through a magnetic field.
The behavior is an observation of the quantization of spin angular momentum and a measurement of the Bohr magneton $\mu_{B}$.
This experiment was first performed in the 1920's and the theoretical foundations, as well as historical accounts, of the experiment can be found in many graduate level quantum mechanics textbooks.

The experimental system consists of a vacuum pump connected to a T-piece pipe.
One side of this pump connects to a housing for an oven.
The oven is, electrically, connected to a DC power supply to control the heating of Poatassium that is place inside the oven.
The other side of the T-pipe connects to a magnet that is in a shape to produce an inhomogenuous magnetic field which interacts with the Potassium atoms ejected from the oven to cause the diffraction of particles.
The magnet sits in between two solenoids that are connected to a DC power supply which produces the current to control the strength of the field.
The magnet is then connect to a flex pipe, which is followed by another piece of straight pipe with a tungsten wire detector connected at its end.
The flex pipe allows for the detector wire to be moved horizontally to measure potassium atoms diffracted by the magnetic field.
The amount of horizontal movement is controlled by a graduated knob attached to a screw above the t-pipe connected to the straight pipe on the otherside of the flex pipe.


The detector wire has three connections.
Two connections allow the detector drive current to pass through connecting a step down transformer to the detector to the ammeter.
This allows the wire to be active for data collection.
The final connection is a BNC connector that allows the signal current measured by the wire to be sent to an amplifier.
The DC current from the detector is amplified from pA level to mA level.
Lastly, there is a power supply that supplies the input AC current into the transformer to drive the tungsten wire detector.

There are seven analog readings that students take to monitor the status of the aparatus and measure the diffraction of the potassium atoms.
Students measure three values at the oven housing.
They measure the oven current, oven voltage, and the oven temperature.
The oven temperature is measured from a type-J thermocouple as a voltage converted to a temperature value.
The fourth measurement is the coil DC current to control the magnetic field.
The last three measurements are taken at the detector wire.
The students measure AC current through the detector wire to confirm the that the detector wire is operating properly.
The input AC voltage at the transformer is measured as a secondary check to confirm that the detector wire is in proper operating conditions.
The final measurement is the signal current from the detector coming out of the amplifier.

To collect data in this experiment students would make sure that there is potassium inside the oven and the pump has evacuated the system to 4.0 x 10$^{-6}$~mbar, or lower.
Once this has been achieved, students turn on the oven current to heat the oven to a temperature between 140\textdegree~C - 170\textdegree~C.
Once this temperature has been achieved the students then confirm that the detector wire current is operating between 4.0 and 4.3 A and that the pressure is still within operating values.
To begin, students begin with the magnetic field off (no current through the solenoids) and thus no diffraction should occur.

After all of this setup, students then rotate the knob on top of the aparatus to determine where the center of the beam is.
Once the students know where the center of the potassium beam is, they then turn on the magnetic field with the coil current to a value between 0.5 and 2.0 A.
The stronger the magnetic field strength, the stronger the diffraction of the potassium atoms.
The atoms are primarily governed by the spin of its electron in the outer-most orbital.
Thus, the students see that the potassium beam splits into two.
Adjusting the knob after this, students see that the splitting is equidistant, on opposite sides, from the center.

The primary reason for updating the setup was to replace the large number of analog readings that the students must make.


