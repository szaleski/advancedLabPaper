\documentclass[10pt, twocolumn, a4paper]{article}
\usepackage[T1]{fontenc}
\usepackage{newtxmath,newtxtext}
\usepackage{hyperref}
\usepackage{graphicx}
\usepackage{indentfirst}
\usepackage{setspace}
\usepackage{xcolor}
\usepackage[export]{adjustbox}
\usepackage{subfig}
\usepackage{authblk}
\setlength{\parskip}{1em}


\topmargin=0.0in
\oddsidemargin=0.0in
\evensidemargin=0in
\textwidth=6.5in
\marginparwidth=0.5in
\headheight=0pt
\headsep=0pt
\textheight=9.0in

\title{\textbf{Modern Techniques to Engage Advanced Laboratory Students using MicroController Hardware and Python}}


%%\author[1]{Thomas Hebbeker}
\author[1]{Kerstin Hoepfner}
\author[1]{Shawn Zaleski$^\dagger$}
%%\author[1]{Henning Keller}
%%\author[1]{Giovanni Mocellin}
\affil[1]{III Physikalisches Institut A RWTH Aachen University}
%%\institute{RWTH-Aachen}
%%\institute{III Physikalisches Institut A RWTH Aachen University}
\date{\empty}
\pagestyle{empty}

\renewcommand{\thesection}{\Roman{section}}

\begin{document}
\twocolumn[
  \begin{@twocolumnfalse}
\maketitle
%%\centerline{Gas Electron Chambers in University Advanced Physics Laboratory}
\noindent
%%\line(1,0){470}
\newline

\begin{abstract}
Engaging students in a physics lab setting can be challenging, especially while modernizing the experiments currently available. Often, students complain about out-dated equipment or ``cookie cutter" type experiments that do not allow much creative freedom on the part of the student. The advanced undergraduate lab course experiments for particle physics at RWTH Aachen University were updated using inexpensive DIY components such as Arduino and Raspberry Pi. To engage students better using Arduino and Python, coding can be left completely to the students to write or they may be supplied with partial, or complete, working programs. This allows the instructor to tailor the lab to the appropriate skill level of the students. We outline two different implementations in lab experiments. One where the Raspberry Pi and Arduino collect Geiger-Mueller data, and another that allows them to collect data with the Arduino via Bluetooth on their smartphone. Student feedback will be presented.

\end{abstract}
  \end{@twocolumnfalse}
  \newpage
]

\section{Introduction}
The Bachelor's advanced physics laboratory at RWTH Aachen University is offered annually to upper division physics students, typically in their final year of physics study.
This laboratory serves roughly 150 students during each offering.
The laboratory is structured largely into two halves.
During the first half of the semester students attend a series of 12 lectures on high energy physics (HEP) detector topics and 12 lectures on solid state physics (SSP)detector topics.
During the second half of the semester students, in pairs of groups of 2 or three students each, perform 5-7 experiments.
These experiments are selected from a set of approximately 25 available experiments of which roughly half are from HEP topics and the other half from SSP topics.
The experiments that the students perform range from classical large-scale experiments such as the Stern-Gerlach experiment to more complex and modern topics such as the gas electron multiplier experiment.


The structure of the experimental portion of each laboratory experiment is broken into three primary parts: discussion, experiment, and report.
Students are expected to have attended the appropriate lecture during the first half of the class as well as to have read the experiment manual prior to performing the experiment.
As such, students are to respond verbally to questions pertaining to the physics theory and electronics hardware operation outlined in the manual and lecture.
An experiment supervisor asks the students the discussion questions in a free response format.
If the students do not demonstrate the minimum level of understanding required to safely perform the experiment, the students are asked to return after reviewing the material further, at which point they are questioned again.
The experiment supervisor then explains, and if necessary demonstrates, any part of the experiment that might be easily damaged, or require special care not described in the manual.
From this point, the students are left to perform the experiment and collect data as outlined in the lab manual.
If the students have questions, encounter problems, or otherwise need to speak to the supervisor, they meet with him as needed.
Lastly, within two weeks of completing the experiment, the students are expected to submit a formal lab report which should take a form similar to that of a typical physics paper.

This paper details some of improvements that were implemented into the advanced physics laboratory course.
In section 2 we motivate these improvements.
In sections 3, 4, and 5, we give detailed descriptions of the portion of experiments that were modified, discussing why the portions were modified and what changes were implemented.
In section 6 we discuss some conclusions found and general student feedback collected following implementation.

\section{Motivation}
As time marches ever onward, existing courses must adapt, change, and improve to better serve students.
Often, students complain that university physics lab experiments use outdated, or even obsolete, technology, that they are not engaged by the experiments, or that the topics covered by the experiments are not applicable in their research.
RWTH Aachen University has identified the above three items to address in an attempt to provide a better laboratory experience for students that utilizes popular modern hardware that engages students in a way that could likely be used by the students in their own research later on.
To this end, three experiments were identified where modern improvements should be attempted:
\begin{enumerate}
\item Gas Detectors and Statistics: Statistics
\item Detector Principles: Multiple-scattering
\item Stern-Gerlach
  
\end{enumerate}

In all of the above experiments, it was decided to replace part of the experiment with modern data acquisition consisting of Arduino micro-controllers, Raspberry Pis, and/or Bluetooth modules.
The hardware was chosen for a variety of reasons, primarily that they are all inexpensive, easy to acquire and use, and there is significant online support for each.
An additional consideration was that, with most of the hardware being popular among electronics hobbyists, it is likely (though not expected) that students have some prior experience with any of the new hardware.


In experiment 1, a Raspberry Pi serving an Arduino that connects to an Arduino compatible PIN photo-diode Geiger-Mueller (GM) counter replaces a bulky, more traditional, gas GM counter powered by a NIM crate with amplifier and seven-segment display counter.
In experiment 2, a similar setup to experiment 1 that utilizes four Arduino compatible gas GM counters, rather than a PIN GM, replaces a traditional gas GM.
In experiment 3, five analog meter readings are replaced by Arduino circuits that send their data via Bluetooth (BT) to a BT capable device, either a laptop or a mobile smart-phone.
We detail these in the coming sections.

\section{Gas Detector and Statistics}
The Statistics portion of the Gas Detector and Statistics experiment was the first experiment in which the new microcontroller technology was implemented.
In this experiment students measure the number of counts registered by a GM counter using a fixed time window of the order of 1-10 $s$ depending on the strength of the source used.
The number of decays from a Sr-90 source and the number of incident cosmic ray muons registered as counts in the time window are used to generate histograms.
The histograms form a well defined (for sufficient number of collections) Gaussian distribution in the case of beta decays, and Poisson distribution in the case of the cosmic rays.
The students then perform a statistical analysis consisting of finding the mean and standard deviation using standard statistical procdures as well as invoking a $\chi^2$-based analysis.

The primary reason for updating this setup is that the traditional gas GM that was used no longer functioned properly and was expensive to replace.
Another reason to move to an updated data acquisition setup was that the NIM crate and supporting modules used to power the GM and register counts required students to record the data using paper and pencil.
Students have complained that this mode of data acquistion is out-dated, typically stating that many modern experiments use some form of computer, or other electronics, based data acquistion and logging.


In an attempt to address these issues, it was decided to use a PIN photodiode GM circuit available from RadiationWatch that is compatible with an Arduino Nano, and a Raspberry Pi, which we will simply refer to as Pi from here.
The GM is a roughly 1~$cm$ x 1~$cm$ with a thin copper shielding wrapped around it.
The GM sits on a PCB having four connection pins.
The Arduino 5~$V$ pin is connected to the V$_{CC}$ pin and the ground pins connect to each.
These supply power to the GM circuit.
The GM signal and noise pins each connect to a digital input pin on the Nano.
These pins send digital HIGH/LOW information from the GM counting circuit.
The Arduino then connects to the Pi via a USB connection.
The Pi acts as the data acquisition control and data storage.
Students can also perform simple analysis since the Pi has Mathematica pre-installed on it.

Except for the USB connections between the Pi and the Arduino, students are not expected to make the physical connections between the detector hardware.
The Arduino and GM sit in pre-fabricated PCBs designed specifically to sit within an aluminum shielding.
The GM PCB is held in place by a 3D printed structure connected to a copper pipe.
The copper pipe aligns the connection between the GM and the Sr-90 source.
The pipe also collimates the beta source to protect the Arduino from unnecessary exposure to beta radiation.

The focus of the experiment is, primarily, on writing code for a set of programs to acquire and store the data followed by a statistical analysis of the data collected.
The students are required to write two pieces of code.
One for controlling the Arduino data acquisition from the GM.
The other is a simple Python program to read and record all of the data transmitted via the Arduino serial port.
Python was chosen as the working programming language as an overwehlmingly large majority of students had, at least, some prior experience working with it.

Students were not expected to have any prior experience working with Arduino microcontrollers, though some did.
The code was written using the Arduino integrated development environment (IDE).
The structure of the code was meant to be simple.
Within this code students were expected to :

\begin{enumerate}
\item Define signal pin and flag variables and counting variables.
\item Begin the serial connection process for the Arduino.
\item Initialize the signal pin as an Arduino input pin and set it as digital high (the GM is active low).
\item Write a data collection loop as follows:
  \begin{enumerate}
  \item Perform digital read on signal pin
  \item Check if signal pin and flag are both low
    \begin{itemize}
    \item If both are low, increment signal count and set flag high (this is done to not double count since the signal pulse lasts multiple loops).
    \item If both are high, reset the flag to low to open another data collection window.
    \end{itemize}
  \item Write a simple `if' statement to control the collection timing.
  \item Send the data collected within the time window via the serial connection.
  \end{enumerate}
  
      
\end{enumerate}

This code was quite simple for students to understand the flow, though the Arduino IDE utilizes a C-like language.
Outside of some dfficulty understanding the signal pulse width relation to collection time and why the signals are digital low, there were no other difficulties for the students.
Most students were able to quickly and easily create this code.

Students were expected to have a little stonger understanding of Python.
The Python script was still expected to be simple.
The structure of the Python code was expected to be:
\begin{enumerate}
\item Import the PySerial library (this allows communication with the serial USB port within the Python program)
\item Create an ouput .txt file to store data
\item Open a serial connection with the Arduino USB port with the proper baud rate.
\item Within the data collection loop:
  \begin{enumerate}
  \item Read the serial data as a list.
  \item Parse the data list and store the necessary variables (signal count and time interval information).
  \item Write this data to file.
    
  \end{enumerate}
\item Write the file to disk for later analysis.
  
\end{enumerate}

Students tended to have more difficulty with this part of the coding, mostly on the debugging side.
Even so, most students were able to quickly and easily write this part of the code.
Once both of these programs are ready students are then able to start with the data collection.

The students then do the following:
\begin{enumerate}
\item Confirm that the USB connection between the Pi and the Arduino is connected.
\item Place the Sr-90 source in the source holder and confirm that it is closed properly.
\item Compile and load the Arduino program to the Arduino.
\item Run the Python program to collect the data.
\item Analyze the data collected.
  
\end{enumerate}

Students then remove the Sr-90 source and reposition the GM apparatus to collect cosmic ray muon data.
The students then anlyzed the data by fitting it to either a Gaussian or a Poisson distribution depending on the source of radiation.
The information from these fits is then used to fit the data to a $\chi^{2}$ distribution to determine a goodness of fit.


In the other half of this experiment, students perform the same experiment using a Cobra3 gas GM from PHYWE.
This setup utilizes it's own pre-defined software to collect data.
As the focus of this paper is on the student based interaction with the new Arduino and Pi hardware and the Python programming, we will not elaborate on the Cobra3 part.
Students apply the same analysis technique to this data as well.
They, then, qualitatively compare the results of the PIN GM with that of the gas GM.

Students responded well to this experiment.

\section{Detector Principles}
The Detector Principles experiment was the next experiment that received an update using new micro-controller technology.
In this experiment, students perform three minor mini experiments to gain some understanding of how particles are detected.
These mini experiments are:
\begin{itemize}
\item Measuring the momentum of charged ($\beta$) particles in a magnetic field
\item Measuring the energy loss of radiation passing through material
\item Multiple scattering measurement of charged ($\beta$) particles passing through an absorber
\end{itemize}

In the first experiment student pass $\beta$ particles through a magnet and measure the momentum through a bending angle measurement.
In the second experiment students place different absorbers in between different radiation sources to measure the energy loss of different forms of radiation through different types of materials.
In the final experiment, students measure the angular dependence of $\beta$ particles from the same source but with different materials placed in between the source and the detector.
It is this last experiment in which the micro-controller technology was introduced.

The procedure that students followed for the multiple scattering experiment is described here.
Students place a cylindrically shaped Sr-90 $\beta$ particle source at the center of a circular arc of 180\textdegree.
This source is placed within an aluminum collimator such that the particles escaping the collimation are those from one of the circular ends of the cylinder.
The circular arc is graduated in 1\textdegree increments.
A Cobra3 gas GM, from PHYWE, is placed at the 0\textdegree mark on the circular arc.
This corresponds to the plane of the GM active area's surface being normal to the collimation of the source.
The source is $\mathcal{O}(40~cm)$ from the detector.
This allows for an angularly dispersed distribution of particles arriving at the detector.
Initially, no absorber is used.
The GM is allowed to recorded for some time $\mathcal{O}(5-10 min)$.
The students then move the GM by 5-10\textdegree and take another reading of same duration as the previous measurement.
The students then take measurements at these intervals for all 5/10\textdegree intervals between -60\textdegree and +60\textdegree.
They then place an absorber and repeat the process for four additional absorbers.
Once the students are done collecting the data, they then plot the values collected as a function of the angle of the GM relative to the center of collimation of the particle source.


The primary reasons to update this setup was to decrease data collection time for this portion of the experiment as well as utilize new technology to better engage students.
Students had similar complaints as with the Gas Detector and Statistics experiment, namely that the data acquisition was becoming out-dated.


To address these issues, it was decided to utilize Arduino and Raspberry Pi hardware as well as utilize several Arduino compatible gas GMs from MightyOhm.
The GM detector consists of an SBM-22 GM counter mounted on a simple digital counting circuit and PCB.
It was decided to have four GM detectors spaced 10\textdegree apart from one another on a circle centered on a Sr-90 $\beta$ particle source.
Each GM was oriented with its longitudinal axis vertical and normal to the particle beam direction.
A structure was engineered in such a way to keep the keep the GM counters centered directly above the angular graduations along the circular segment that was used previously.
This offers the ability for students to reduce the duration of the experiment, collect more data in the same amount of time, or a combination of the two.


Students connect each GM circuit to the same Arduino Uno.
The Arduino multiplexes the signal information from each GM counter.
The Arduino is connected via USB to the Raspberry Pi, on which the the data is stored.
The students do not write their own Arduino sketch, nor do they write their own data acquisition code on the Raspberry Pi.
Due to time constraints from the other two mini experiments, students are only expected to move the GMs and run simple terminal commands using software written by the laboratory staff that are known to work properly.

\section{Stern-Gerlach Experiment}

The final experiment that was updated with microcontroller technollogy was the Stern-Gerlach experiment.
In this experiment, students investigate the diffraction effects exhibited on particles passing through a magnetic field.
The behavior is an observation of the quantization of spin angular momentum and a measurement of the Bohr magneton $\mu_{B}$.
This experiment was first performed in the 1920's and the theoretical foundations, as well as historical accounts, of the experiment can be found in many graduate level quantum mechanics textbooks.

The experimental system consists of a vacuum pump connected to a T-piece pipe.
One side of this pump connects to a housing for an oven.
The oven is, electrically, connected to a DC power supply to control the heating of Poatassium that is place inside the oven.
The other side of the T-pipe connects to a magnet that is in a shape to produce an inhomogenuous magnetic field which interacts with the Potassium atoms ejected from the oven to cause the diffraction of particles.
The magnet sits in between two solenoids that are connected to a DC power supply which produces the current to control the strength of the field.
The magnet is then connect to a flex pipe, which is followed by another piece of straight pipe with a tungsten wire detector connected at its end.
The flex pipe allows for the detector wire to be moved horizontally to measure potassium atoms diffracted by the magnetic field.
The amount of horizontal movement is controlled by a graduated knob attached to a screw above the t-pipe connected to the straight pipe on the otherside of the flex pipe.


The detector wire has three connections.
Two connections allow the detector drive current to pass through connecting a step down transformer to the detector to the ammeter.
This allows the wire to be active for data collection.
The final connection is a BNC connector that allows the signal current measured by the wire to be sent to an amplifier.
The DC current from the detector is amplified from pA level to mA level.
Lastly, there is a power supply that supplies the input AC current into the transformer to drive the tungsten wire detector.

There are seven analog readings that students take to monitor the status of the aparatus and measure the diffraction of the potassium atoms.
Students measure three values at the oven housing.
They measure the oven current, oven voltage, and the oven temperature.
The oven temperature is measured from a type-J thermocouple as a voltage converted to a temperature value.
The fourth measurement is the coil DC current to control the magnetic field.
The last three measurements are taken at the detector wire.
The students measure AC current through the detector wire to confirm the that the detector wire is operating properly.
The input AC voltage at the transformer is measured as a secondary check to confirm that the detector wire is in proper operating conditions.
The final measurement is the signal current from the detector coming out of the amplifier.

To collect data in this experiment students would make sure that there is potassium inside the oven and the pump has evacuated the system to 4.0 x 10$^{-6}$~mbar, or lower.
Once this has been achieved, students turn on the oven current to heat the oven to a temperature between 140\textdegree~C - 170\textdegree~C.
Once this temperature has been achieved the students then confirm that the detector wire current is operating between 4.0 and 4.3 A and that the pressure is still within operating values.
To begin, students begin with the magnetic field off (no current through the solenoids) and thus no diffraction should occur.

After all of this setup, students then rotate the knob on top of the aparatus to determine where the center of the beam is.
Once the students know where the center of the potassium beam is, they then turn on the magnetic field with the coil current to a value between 0.5 and 2.0 A.
The stronger the magnetic field strength, the stronger the diffraction of the potassium atoms.
The atoms are primarily governed by the spin of its electron in the outer-most orbital.
Thus, the students see that the potassium beam splits into two.
Adjusting the knob after this, students see that the splitting is equidistant, on opposite sides, from the center.

The primary reason for updating the setup was to replace the large number of analog readings that the students must make with modern microcontroller readout.
To replace these signals, Arduino Unos and simple circuits were used to collect these readings.
It was initially designed such that students would be able to use their smartphones to read the values collected by the Arduinos.

There are, essentially, three general different circuit setups that are used, however some features are shared in common with each circuit.
These commonalities will be discussed briefly before discussing how each of the three circuits differs.
Initially, each measurement has an Arduino Uno powered by a 12 V at 1.67 A DC power supply.
The 5~V and GND pins, as well as the SCL and SDA connections were connected to an ADS1115 12 bit gain amplifier analog to digital converter (ADC).
The ADS1115 was chosen because it has 6 different gain settings available that allowed some flexibility in precision of the data measurement.
The same pins that connected the Arduino to the ADS1115 connected to an HC-05 serial bluetooth module.
Except in one circuit case, one of the four ADS1115 analog input pins and the GND pin connect across a resistor, or voltage divider, depending on the particular circuit.
The ADS1115 converts this analog voltage reading into a digital signal then sends this information to the Arduino.
The Arduino then performs a calculation to convert the digital reading into the appropriate current or voltage reading, again depending on the particular circuit.
The Arduino then sends this information via the Serial line to the bluetooth module.
Once the students have paired their device to the bluetooth module, and downloaded the free Android app, are able to read the output from the Arduino on their smartphone.
Due to reasons that will become clear when discussing student feedback in the next section, the bluetooth modules and the power supplies were replaced with a USB connector.
Students then read all of their data to file using a Python program.

In the special case of the temperature reading, the voltage that is read from the thermocouple is in the 0-10~mV range.
As such a small voltage value is often observed, the output lead were directly connected to the analog in and the GND connections of the ADS1115.
This was the simplest circuit.

In the case of current measurement, current sense resistors with resistances between 1-500~$m\Omega$ were used.
The analog input of the ADS1115 connect to one side of the resistor and the other leg of the resistor connected to the GND pin.
The small resistance values were chosen so to, as minimally as possible, avoid changing the optimum values flowing through the system.

In the case of the larger voltage from the power supply, theese voltages typically exceed the maximum voltage which the ADS1115 and the Arduino can withstand.
As such, a voltage divider  with voltage values of 3.33~$k\Omega$ and 6.67~$k\Omega$ were chosen to limit current and reduce the voltages to be within a tolerable range for the microcontroller.
The ADS1115 measured the voltage across the 3.33~$k\Omega$ resistor.

\section{Student Response to the New Hardware and Experimental Procedure}
After making these changes, we found the following from the students...
The changes to the advanced lab experiments outlined in the previous three sections were implemented in the previous Winter 2019 semester.
At the end of each of the experiments the students were given a short questionnaire related to the changes made to the specific experiment that they performed.
The focus of the questionnaire was to understand if there were any perceived improvements from the students' perspective or if they benefitted in some way from the changes.
The questionnaire included questions related to the students' previous experience working with the microcontroller hardware and programming.
It also included questions to guage if the students enjoyed, or felt some benefit from performing the experiment.
The last portion of the exam was free response and asked the students to identify what they enjoyed about the experiment as well as what they thought could be further improved.

The questionnaires were collected and recorded promptly after they were answered by the students.
This was done in order to adjust the lab experiment quickly and efficiently if there were issues using the equipment, or if a portion of the procedure was not working well.
We present and discuss the results of the student response to these changes to the lab experiments.
Where appropriate, we also discuss any changes that were made in the middle of the semester for these experiments.

\subsection{Gas Detectors and Statistics}
The first two items in this questionnaire asked about the students' prior experience with Arduino+Raspberry Pi microcontrollers and Python programming.
These questions were only intended to get a coarse guage of the students' command and level of experience with these items.
12 students responded that they had no prior experience with Arduino and/or Raspberry Pi, while three responded that they had 'some' prior experience with these microcontrollers.
Here, some is defined to be less than three years of experience.
There were two other oprtions that students could have selected: 'significant' or 'expert'.
Significant was defined to be more than 3 years of experience, while expert would be considered as a fluent user of these items.
As a reference, 16 students responded to the questionnaire, however, not all students responded to all questions.
This suggests that the students that participated in this experiment did not have, collectively, much experience working with these microcontrollers prior to taking the course.

Students had much more experience using Python programming language.
Only a single student responded that they had no prior experience with Python.
13 students responded that they had 'some' experience.
One student answered that they had 'significant' experience.
No students considered themselves experts at writing Python code.
The degrees of this question were the same as the previous.
The student repsonses to these questions are shown in Tab. 1.



Questions three and four asked if the students enjoyed using the Raspberry Pi and the Arduino, respectively.
The goal of these questions was to understand if the students enjoyed using the microcontrollers and, from this, infer if future students might enjoy working with them as well.
The results for these questions are displayed in Tab. 2.
9 students agreed, or strongly agreed, that they enjoyed using the Raspberry Pi while 5 responded neutrally.
12 students agreed, or strongly agreed, that they enjoyed using the Arduino, while only a single student disliked working with it.
Overall, this would suggest that the students that performed this experiment enjoyed working with the microcontrollers and would infer that future students are likely to enjoy using them as well.

Question five tried to determine the students felt that working with the combination of Raspberry Pi, Arduino, and Python programming made the experiment more interesting when compared to other experiments within the lab course.
These results are shown in Tab. 3.
13 students agreed, or strongly agreed, that the combination of new technologies made for a more interesting experiment to work with.
Only one student remained neutral on this response.

Questions six, seven, and eight asked the students to evaluate, qualitatively, how much they felt that they learned about working with the Raspberry Pi, Arduino, and Python, respectively.
The responses to each of these questions can be found in Tab. 4.
Eight students reported that they gained some to alot knowledge about working with the Raspberry Pi.
Four student felt that they learned very little, or nothing about working with the Raspberry Pi.
Two students responded with a 'not applicable' answer.
12 students learned some to alot of new information about working with the Arduino, while two students felt that they learned very little, or nothing.
In response to the item on Python, 4 students responded that they learned some to alot of new items realted to programming.
Nine students responded that they learned nothing, or very little about Python.
One student responded with 'not applicable'.
As expected from the earlier responses, students felt that they learned more about the microcontrollers than they did from working with Python.


Students were given a chance to elaborate more on the experiment with free responses to what they liked about working with the new technology, what they didn't like about it, and any suggestions for improvement moving forward.
The free responses to what they enjoyed about the experiment fit into seven different categories.
There were a total of 15 responses to this portion.
The first category to be discussed is that of students enjoying solving a complex problem (at least partially) on their own.
Two students responded in a vein similar to ``I enjoyed writing my own computer program to perform this experiment''.
These students appreciated being able to write their own Python and/or Arduino program to collect and store data to solve an experimental problem that they were given.
A related category is using 'old' knowledge to solve a new problem.
Here old knowledge simply refers to that which the student already had prior to the start of the experiment.
This student responded that applying their programming knowledge to solve a physics problem was what they liked about the experiment.

Three students repsonded that they liked the choice of programming language(s) which they had to utilize to perform the experiment.
These students either liked that they enjoyed working with a language that they were already familiar with, learning a new language, or even working with two different languages within the same experimental framework.
Four students reported that they simply enjoyed learning very generally about working with the Raspberry Pi or Arduino microcontrollers.
One student stated that they liked that the microcontrollers were small and could be used for many different applications.

Two students liked that the new microcontroller technology was easy to use.
Two students also found that using the new technology made understanding how data is measured and recorded more apparent.

The students were also asked about what they didn't like about the experiment.
There were 15 responses to this item that fit into four different general categories.

The first category, receiving by far the largest number of complaints, is related to experiment logistics.
Eight students complained about either not having enough space for the experiment (4), having to switch the monitor connection between the Windows PC and the Raspberry Pi (3), or having a US keyboard layout rather than German (1).
These complaints were addressed within the first few lab sessions to provide more space, an additional monitor so that each computer has it's own, and to provide a German keyboard if they would want to use it.

The second category is related to computer program debugging.
Four students report that they were unable to debug their Arduino or Python program within the alloted time of the experiment without help from the experiment supervisor.
It was difficult to directly address this particular item in the middle of the semester.
Moving forward, it would be best that the instructions recommend that the students attempt to write the programs prior to starting the experiment or to supply the students with a known working software.
Though, the latter would defeat the purpose of allowing them to gain the experience of writing their own code.

Two students suggested that the experiment manual could have better descriptions of the some of the Python libraries and data analysis.
This material will be improved in the second iteration of the lab manual.
And one student complained that the Raspberry Pi was not necessary for the experiment.
The authors' response to this particular item is that it is true that one can collect the data with the GM counter, and Arduino only, using an Arduino compatible data-logging shield.
However, one of the intentions of the upgrade experiment is to expose the students to multiple microcontroller technologies.
Many experimenters, scientists, and electronics enthusiasts use the Raspberry Pi as a single-board PC.
It was simply envisioned to be used as an inexpensive computer to store and analyze data.
We respect the student's astute observation, but note that it misses one of the learning target items.

In the final item that the students were asked to respond to, students were allowed to propose suggestions to the experiment for the next iteration.
All of the responses repeat those from the previous two items. Thus we will not elaborate further on them.

We learned some important lessons to guide us in developing microcontroller improvements from this particular questionnaire.
An overwhelming percentage of students came in with little, or no, experience working with microcontrollers.
However, the same students enter the lab with sufficient programming experience that they can, with a guided manual, develop their own data storage code using a programming language that they are already comfortable with.
They can also develop a simple working data acquisition program using a program that they have not used before.
The students responded that they enjoy this as well as see it as an improved way for them to collect data.
From this we can say, at least for this small sample size, that this new microcontroller technology presented in a user-friendly way is an improved way to engage students in a university laboratory setting.

\subsection{Detector Principles}
Owing to the similiarity of hardware between the Gas Detector and Statistics lab and this one, many of the questions asked to the students are similar.
No Python programming was expected, thus some of the Python questions that occur in the previous discussion are substituted for similar ones regarding the Arduino compatible GM counter.
For reference 17 students responded to the questionnaire, but not all students responded to all of the questions.

In the first question, we ask the students how much prior combined experience they had working with Arduino and Raspberry Pi microcontrollers.
11 students responded that they had no experience with Arduino or Raspberry Pi before starting the experiment.
Six student answered that they had 'some' prior experience with these microcontrollers.
No students responded that they had 'significant' or 'expert' experience.
These responses have the same designation as the first two questions in the questionnaire related to the Gas Detector and Statistics experiment.
These responses are shown in Tab. 5.

Questions two and three are the same as three and four from the Gas Detector and Statistics questionnaire.
13 students agreed, or strongly agreed, that they enjoyed working with the Raspberry Pi while two students were neutral and one responded with 'not applicable'.
13 students agreed, or strongly agreed, to enjoying using the Arduino.
One student was neutral and three responded with 'not applicable.
These results are shown in Tab. 6.

Question four asks the students to respond to how interesting they found the new technology.
These results can be found in Tab. 7.
11 students agreed, or strongly agreed, that the new combination of the Raspberry Pi, Arduino, and GM counter made the experiment more interesting.
Two students disagreed, while three students responded neutrally, and one responded with 'not applicable'.

Questions five and six ask the students to evaluate how much the felt that they learned by working with the Raspberry Pi and Arduino, respectively.

\section{Summary}
In summary, three experiments, or portions thereof, were modified using microcontroller technology coupled with either Python programming or provisions for reading data via smartphones.
In the Gas Detector and Statistics experiment, an Arduino compatible PIN GM was used in conjunction with an Arduino Nano and Raspberry Pi.
Students were expected to write their own Arduino data acquisition code and Python data storage code.
Students, overall, generally enjoyed this experiment and felt that the inclusion of the new technology helped to make the experiment more interesting.
Students particularly felt that they got extra experience, and one could infer, increased self sufficiency working with this experiment.
Students felt that there was more transparency in how the data was collected, stored, and analyzed compared to a standard experiment that operates as more of a black box.
In short, there are only minor improvements that we would make to this lab.
Namely, in finding better ways to present material in the lab manual.
Another improvement would be to, possibly, allow students to start writing code prior to the experiment to allow sufficient time for debugging.

In the Mulitple Scattering portion of the Detector Principles experiment, a single, more traditional, gas GM was replaced by a series of four Arduino compatible gas GMs controlled by an Arduino Uno with data stored on a Raspberry Pi.
In this experiment, the goal of modernization was to simply reduce the data collection time by increasing the number of data points collected simulataneously using modern microcontroller technology.
Another goal was to find a better way to store the data using the Raspberry Pi.
Students had a little less freedom to develop their own programs in this experiment compared to that of the Gas Detector and Statistics experiment.
This was intentional as the Multiple Scattering experiment comprises one third of the amount of experiments in the Detector Principles lab, which is already typically running short on allotted time.
Students, overall, appreciated the efforts of modernizing this lab.
They liked that there were new technology that they had the opportunity to try out.
The students responded that they would like to have further automatization of the data collection as well as controlling the angular position of the detectors using the Arduino.
Controlling the angle would be straight-forward using a stepper motor, or something similar.
The automation of the data collection could be done using Python scripting, or something similar by the students.
In short, there are some minor improvements that can be done to improve the experiment for the students.

In the Stern-Gerlach, the third and final experiment which we added Arduino technology to, we simply replaced all of the analog readouts with Bluetooth serial readouts.
Thus, students would be able to use their own smartphones with Android serial terminal applications to read the data.
Several issues were found with this mode of data acquisition.
The most important of these being that the Apple iStore does not support a free application that is compatible with the experiment for students with Apple devices.
Secondarily, some students do not have the ability to keep their devices charged.
Thirdly, students could only listen to a single bluetooth+Arduino combination.
Lastly, there was not a good way of storing the data in a way that students could utilize it.

After several attempts using the smartphone readout, it was abandoned in favor of USB to laptop readout.
This worked better, solving the issue with the iOS app and storing data.
However, some students still had difficulty because they often had an insufficient number of USB ports for the experiment.
In addition, even if the students had sufficient USB ports to accomadate the large number of Arduinos used in the experiment, they were still unable to read data from more than one Arduino at the same time.

Students, overall, generally liked the type of technology that was being introduced into this experiment.
However, students would have appreciated the experiment changes that were made if they worked with little to no difficulty.
Additionally, students voiced that they would like more freedom to make changes to the coding and data acquisition process.
Most of this can be accomplished easily by introducing a Raspberry Pi and reducing the number of Arduinos needed to perform all of the measurements required for this laboratory.

All of these changes that we made to the laboratory experiments were made in an attempt to modernize them and engage students in a more positive way.
Based on the feedback from the students, we would say that we were, overall, successful in doing so.
We also understand that there are certain cases where some, not insignificant, improvement is necessary prior to the next iteration of the laboratory course.
However, thanks to the student feedback we know where most of the important focus points for the discussed experiments are.
Also, thanks to the overwhelmingly positive feedback, we also know that students enjoy working with this type of equipment and enjoy being given more freedom to develop theor own solutions to collecting and analyzing the data in most experiments.
Thus, this gives a starting point for not only improving other existing experiments within the course, but also in designing new experiments moving forward.



\end{document}
