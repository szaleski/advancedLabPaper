\documentclass[12pt]{article}
\usepackage{hyperref}
\usepackage{graphicx}
\usepackage{indentfirst}
\usepackage{setspace}
\usepackage{xcolor}
\usepackage[export]{adjustbox}
\usepackage{subfig}
\usepackage{authblk}
\setlength{\parskip}{1em}


\topmargin=0.0in
\oddsidemargin=0.0in
\evensidemargin=0in
\textwidth=6.5in
\marginparwidth=0.5in
\headheight=0pt
\headsep=0pt
\textheight=9.0in

\title{\textbf{Modern Techniques to Engage Advanced Laboratory Students using MicroController Hardware and Python}}

\author[1]{Shawn Zaleski}
%%\author[1]{Thomas Hebbeker}
\author[1]{Kerstin Hoepfner}
%%\author[1]{Henning Keller}
%%\author[1]{Giovanni Mocellin}
\affil[1]{III Physikalisches Institut A RWTH Aachen University}
%%\institute{RWTH-Aachen}
%%\institute{III Physikalisches Institut A RWTH Aachen University}
\date{\empty}
\pagestyle{empty}

\begin{document}
\maketitle
%%\centerline{Gas Electron Chambers in University Advanced Physics Laboratory}
\noindent
%%\line(1,0){470}
\newline

\begin{abstract}
Engaging students in a physics lab setting can be challenging, especially while modernizing the experiments currently available. Often, students complain about out-dated equipment or ``cookie cutter" type experiments that do not allow much creative freedom on the part of the student. The advanced undergraduate lab course experiments for particle physics at RWTH Aachen University were updated using inexpensive DIY components such as Arduino and Raspberry Pi. To engage students better using Arduino and Python, coding can be left completely to the students to write or they may be supplied with partial, or complete, working programs. This allows the instructor to tailor the lab to the appropriate skill level of the students. We outline two different implementations in lab experiments. One where the Raspberry Pi and Arduino collect Geiger-Mueller data, and another that allows them to collect data with the Arduino via Bluetooth on their smartphone. Student feedback will be presented.

\end{abstract}

\newpage

\section{Introduction}
In the advanced physics lab...

\section{Motivation}
To address student complaints within the advanced lab...

\section{Gas Detector and Statistics}
An update to the Gas Detectors and Statistics Laboratory....

\section{Detector Principles}
An update to the Multiple Scattering portion of the Detector Principles experiment...

\section{Stern-Gerlach Experiment}
An update to the readout of the Stern-Gerlach Experiment...
The final experiment that was updated with microcontroller technollogy was the Stern-Gerlach experiment.
In this experiment, students investigate the diffraction effects exhibited on particles passing through a magnetic field.
The behavior is an observation of the quantization of spin angular momentum and a measurement of the Bohr magneton $\mu_{B}$.
This experiment was first performed in the 1920's and the theoretical foundations, as well as historical accounts, of the experiment can be found in many graduate level quantum mechanics textbooks.

The experimental system consists of a vacuum pump connected to a T-piece pipe.
One side of this pump connects to a housing for an oven.
The oven is, electrically, connected to a DC power supply to control the heating of Poatassium that is place inside the oven.
The other side of the T-pipe connects to a magnet that is in a shape to produce an inhomogenuous magnetic field which interacts with the Potassium atoms ejected from the oven to cause the diffraction of particles.
The magnet sits in between two solenoids that are connected to a DC power supply which produces the current to control the strength of the field.
The magnet is then connect to a flex pipe, which is followed by another piece of straight pipe with a tungsten wire detector connected at its end.
The flex pipe allows for the detector wire to be moved horizontally to measure potassium atoms diffracted by the magnetic field.
The amount of horizontal movement is controlled by a graduated knob attached to a screw above the t-pipe connected to the straight pipe on the otherside of the flex pipe.


The detector wire has three connections.
Two connections allow the detector drive current to pass through connecting a step down transformer to the detector to the ammeter.
This allows the wire to be active for data collection.
The final connection is a BNC connector that allows the signal current measured by the wire to be sent to an amplifier.
The DC current from the detector is amplified from pA level to mA level.
Lastly, there is a power supply that supplies the input AC current into the transformer to drive the tungsten wire detector.

There are seven analog readings that students take to monitor the status of the aparatus and measure the diffraction of the potassium atoms.
Students measure three values at the oven housing.
They measure the oven current, oven voltage, and the oven temperature.
The oven temperature is measured from a type-J thermocouple as a voltage converted to a temperature value.
The fourth measurement is the coil DC current to control the magnetic field.
The last three measurements are taken at the detector wire.
The students measure AC current through the detector wire to confirm the that the detector wire is operating properly.
The input AC voltage at the transformer is measured as a secondary check to confirm that the detector wire is in proper operating conditions.
The final measurement is the signal current from the detector coming out of the amplifier.

To collect data in this experiment students would make sure that there is potassium inside the oven and the pump has evacuated the system to 4.0 x 10$^{-6}$~mbar, or lower.
Once this has been achieved, students turn on the oven current to heat the oven to a temperature between 140\textdegree~C - 170\textdegree~C.
Once this temperature has been achieved the students then confirm that the detector wire current is operating between 4.0 and 4.3 A and that the pressure is still within operating values.
To begin, students begin with the magnetic field off (no current through the solenoids) and thus no diffraction should occur.

After all of this setup, students then rotate the knob on top of the aparatus to determine where the center of the beam is.
Once the students know where the center of the potassium beam is, they then turn on the magnetic field with the coil current to a value between 0.5 and 2.0 A.
The stronger the magnetic field strength, the stronger the diffraction of the potassium atoms.
The atoms are primarily governed by the spin of its electron in the outer-most orbital.
Thus, the students see that the potassium beam splits into two.
Adjusting the knob after this, students see that the splitting is equidistant, on opposite sides, from the center.

The primary reason for updating the setup was to replace the large number of analog readings that the students must make.



\section{Student Response to the New Hardware and Experimental Procedure}
After making these changes, we found the following from the students...
The changes to the advanced lab experiments outlined in the previous three sections were implemented in the previous Winter 2019 semester.
At the end of each of the experiments the students were given a short questionnaire related to the changes made to the specific experiment that they performed.
The focus of the questionnaire was to understand if there were any perceived improvements from the students' perspective or if they benefited in some way from the changes.
The questionnaire included questions related to the students' previous experience working with the micro-controller hardware and programming.
It also included questions to gauge if the students enjoyed, or felt some benefit from performing the experiment.
The last portion of the exam was free response and asked the students to identify what they enjoyed about the experiment as well as what they thought could be further improved.

The questionnaires were collected and recorded promptly after they were answered by the students.
This was done in order to adjust the lab experiment quickly and efficiently if there were issues using the equipment, or if a portion of the procedure was not working well.
We present and discuss the results of the student response to these changes to the lab experiments.
Where appropriate, we also discuss any changes that were made in the middle of the semester for these experiments.

\subsection{Gas Detectors and Statistics}
The first two items in this questionnaire asked about the students' prior experience with Arduino+Raspberry Pi micro-controllers and Python programming.
These questions were only intended to get a coarse gauge of the students' command and level of experience with these items.
12 students responded that they had no prior experience with Arduino and/or Raspberry Pi, while three responded that they had 'some' prior experience with these micro-controllers.
Here, some is defined to be less than three years of experience.
There were two other options that students could have selected: 'significant' or 'expert'.
Significant was defined to be more than 3 years of experience, while expert would be considered as a fluent user of these items.
As a reference, 16 students responded to the questionnaire, however, not all students responded to all questions.
This suggests that the students that participated in this experiment did not have, collectively, much experience working with these micro-controllers prior to taking the course.

Students had much more experience using Python programming language.
Only a single student responded that they had no prior experience with Python.
13 students responded that they had 'some' experience.
One student answered that they had 'significant' experience.
No students considered themselves experts at writing Python code.
The degrees of this question were the same as the previous.
The student responses to these questions are shown in Tab.\ref{tab:t7q12}.

\begin{table}[htpb]
  %%  \resizebox{\textwidth}{!}{%
  \scriptsize
  \begin{center}
    \caption{\scriptsize Student responses to the first two questions of the Gas Detector and Statistics Questionnaire.}
    \label{tab:t7q12}
    \begin{itemize}
    \item 1) How much experience with Arduino and/or Raspberry Pi equipment did you have prior to this experiment?
    \item 2) How much experience with Python did you have prior to this experiment?
    \end{itemize}
    

    \begin{tabular}{|c | c | c | c | c |}
      \hline
      Subject & none & some ($<$ 3 yr) & significant  ($>$ 3 yr) & expert\\
      \hline
      Arduino+Raspberry Pi & 12 & 3 & 0 & 0\\
      \hline
      Python & 1 & 13 & 1 & 0\\
      \hline
    \end{tabular}
  \end{center}%%}
\end{table}


Questions three and four asked if the students enjoyed using the Raspberry Pi and the Arduino, respectively.
The goal of these questions was to understand if the students enjoyed using the micro-controllers and, from this, infer if future students might enjoy working with them as well.
The results for these questions are displayed in Tab.\ref{tab:t7q34}.
9 students agreed, or strongly agreed, that they enjoyed using the Raspberry Pi while 5 responded neutrally.
12 students agreed, or strongly agreed, that they enjoyed using the Arduino, while only a single student disliked working with it.
Overall, this would suggest that the students that performed this experiment enjoyed working with the microcontrollers and would infer that future students are likely to enjoy using them as well.


\begin{table}[htpb]
  %%  \resizebox{\textwidth}{!}{%
  \scriptsize
  \begin{center}
    \caption{\scriptsize Student responses to the questions three and four of the Gas Detector and Statistics Questionnaire.}
    \label{tab:t7q34}
    \begin{itemize}
    \item 3) Did you enjoy using the Raspberry Pi as the data collection medium?
    \item 4) Did you enjoy working with the Arduino Project?
    \end{itemize}
    

    \begin{tabular}{|c | c | c | c | c | c | c|}
      \hline
      Subject & SA & A & N & D & SD & N/A\\
      \hline
      Raspberry Pi & 3 & 6 & 5 & 0 & 0 & 0\\
      \hline
      Arduino & 5 & 7 & 0 & 1 & 0 & 0\\
      \hline
    \end{tabular}
  \end{center}%%}
\end{table}

Question five tried to determine the students felt that working with the combination of Raspberry Pi, Arduino, and Python programming made the experiment more interesting when compared to other experiments within the lab course.
These results are shown in Tab.\ref{tab:t7q5}.
13 students agreed, or strongly agreed, that the combination of new technologies made for a more interesting experiment to work with.
Only one student remained neutral on this response.

\begin{table}[htpb]
  %%  \resizebox{\textwidth}{!}{%
  \scriptsize
  \begin{center}
    \caption{\scriptsize Student responses to the question five of the Gas Detector and Statistics Questionnaire.}
    \label{tab:t7q5}
    \begin{itemize}
    \item  5) Did the combination of new technology for RaspberryPi/Python/Arduino make the experiment more interesting?
    \end{itemize}
    

    \begin{tabular}{|c | c | c | c | c | c | c|}
      \hline
      Subject & SA & A & N & D & SD & N/A\\
      \hline
      Raspberry Pi/Python/Arduino & 5 & 8 & 1 & 0 & 0 & 0\\
      \hline
    \end{tabular}
  \end{center}%%}
\end{table}

Questions six, seven, and eight asked the students to evaluate, qualitatively, how much they felt that they learned about working with the Raspberry Pi, Arduino, and Python, respectively.
The responses to each of these questions can be found in Tab.\ref{tab:t7q678}.
Eight students reported that they gained some to alot knowledge about working with the Raspberry Pi.
Four student felt that they learned very little, or nothing about working with the Raspberry Pi.
Two students responded with a 'not applicable' answer.
12 students learned some to alot of new information about working with the Arduino, while two students felt that they learned very little, or nothing.
In response to the item on Python, 4 students responded that they learned some to alot of new items related to programming.
Nine students responded that they learned nothing, or very little about Python.
One student responded with 'not applicable'.
As expected from the earlier responses, students felt that they learned more about the micro-controllers than they did from working with Python.


\begin{table}[htpb]
  %%  \resizebox{\textwidth}{!}{%
  \scriptsize
  \begin{center}
    \caption{\scriptsize Student responses to the questions six, seven, and eight of the Gas Detector and Statistics Questionnaire.}
    \label{tab:t7q678}
    \begin{itemize}
    \item 6) How much did you learn about working with Raspberry Pi?
    \item 7) How much did you learn about working with Arduino?
    \item 8) How much did you learn about working with Python?
      
    \end{itemize}
    

    \begin{tabular}{|c | c | c | c | c | c | c|}
      \hline
      Subject & quite alot & some & very little & N/A\\
      \hline
      Raspberry Pi & 2 & 6 & 3 & 1 & 2 \\
      \hline
      Arduino & 4 & 8 & 1 & 1 & 0 \\
      \hline
      Python & 1 & 3 & 8 & 1 & 1 \\
      \hline
    \end{tabular}
  \end{center}%%}
\end{table}





Students were given a chance to elaborate more on the experiment with free responses to what they liked about working with the new technology, what they didn't like about it, and any suggestions for improvement moving forward.
The free responses to what they enjoyed about the experiment fit into seven different categories.
There were a total of 15 responses to this portion.
The first category to be discussed is that of students enjoying solving a complex problem (at least partially) on their own.
Two students responded in a vein similar to ``I enjoyed writing my own computer program to perform this experiment''.
These students appreciated being able to write their own Python and/or Arduino program to collect and store data to solve an experimental problem that they were given.
A related category is using 'old' knowledge to solve a new problem.
Here old knowledge simply refers to that which the student already had prior to the start of the experiment.
This student responded that applying their programming knowledge to solve a physics problem was what they liked about the experiment.

Three students responded that they liked the choice of programming language(s) which they had to utilize to perform the experiment.
These students either liked that they enjoyed working with a language that they were already familiar with, learning a new language, or even working with two different languages within the same experimental framework.
Four students reported that they simply enjoyed learning very generally about working with the Raspberry Pi or Arduino micro-controllers.
One student stated that they liked that the micro-controllers were small and could be used for many different applications.

Two students liked that the new micro-controller technology was easy to use.
Two students also found that using the new technology made understanding how data is measured and recorded more apparent.

The students were also asked about what they didn't like about the experiment.
There were 15 responses to this item that fit into four different general categories.

The first category, receiving by far the largest number of complaints, is related to experiment logistics.
Eight students complained about either not having enough space for the experiment (4), having to switch the monitor connection between the Windows PC and the Raspberry Pi (3), or having a US keyboard layout rather than German (1).
These complaints were addressed within the first few lab sessions to provide more space, an additional monitor so that each computer has it's own, and to provide a German keyboard if they would want to use it.

The second category is related to computer program debugging.
Four students report that they were unable to debug their Arduino or Python program within the allotted time of the experiment without help from the experiment supervisor.
It was difficult to directly address this particular item in the middle of the semester.
Moving forward, it would be best that the instructions recommend that the students attempt to write the programs prior to starting the experiment or to supply the students with a known working software.
Though, the latter would defeat the purpose of allowing them to gain the experience of writing their own code.

Two students suggested that the experiment manual could have better descriptions of the some of the Python libraries and data analysis.
This material will be improved in the second iteration of the lab manual.
And one student complained that the Raspberry Pi was not necessary for the experiment.
The authors' response to this particular item is that it is true that one can collect the data with the GM counter, and Arduino only, using an Arduino compatible data-logging shield.
However, one of the intentions of the upgrade experiment is to expose the students to multiple micro-controller technologies.
Many experimenters, scientists, and electronics enthusiasts use the Raspberry Pi as a single-board PC.
It was simply envisioned to be used as an inexpensive computer to store and analyze data.
We respect the student's astute observation, but note that it misses one of the learning target items.

In the final item that the students were asked to respond to, students were allowed to propose suggestions to the experiment for the next iteration.
All of the responses repeat those from the previous two items. Thus we will not elaborate further on them.

We learned some important lessons to guide us in developing micro-controller improvements from this particular questionnaire.
An overwhelming percentage of students came in with little, or no, experience working with micro-controllers.
However, the same students enter the lab with sufficient programming experience that they can, with a guided manual, develop their own data storage code using a programming language that they are already comfortable with.
They can also develop a simple working data acquisition program using a program that they have not used before.
The students responded that they enjoy this as well as see it as an improved way for them to collect data.
From this we can say, at least for this small sample size, that this new micro-controller technology presented in a user-friendly way is an improved way to engage students in a university laboratory setting.

\subsection{Detector Principles}
Owing to the similarity of hardware between the Gas Detector and Statistics lab and this one, many of the questions asked to the students are similar.
No Python programming was expected, thus some of the Python questions that occur in the previous discussion are substituted for similar ones regarding the Arduino compatible GM counter.
For reference 17 students responded to the questionnaire, but not all students responded to all of the questions.

In the first question, we ask the students how much prior combined experience they had working with Arduino and Raspberry Pi micro-controllers.
11 students responded that they had no experience with Arduino or Raspberry Pi before starting the experiment.
Six student answered that they had 'some' prior experience with these micro-controllers.
No students responded that they had 'significant' or 'expert' experience.
These responses have the same designation as the first two questions in the questionnaire related to the Gas Detector and Statistics experiment.
These responses are shown in Tab.\ref{tab:t12q1}.

\begin{table}[htpb]
  %%  \resizebox{\textwidth}{!}{%
  \scriptsize
  \begin{center}
    \caption{\scriptsize Student responses to the first question of the Detector Principles Questionnaire.}
    \label{tab:t12q1}
    \begin{itemize}
    \item 1) How much experience with Arduino and/or Raspberry Pi equipment did you have prior to this experiment?
    \end{itemize}
    

    \begin{tabular}{|c | c | c | c | c |}
      \hline
      Subject & none & some ($<$ 3 yr) & significant  ($>$ 3 yr) & expert\\
      \hline
      Arduino+Raspberry Pi & 11 & 6 & 0 & 0\\
      \hline
    \end{tabular}
  \end{center}%%}
\end{table}

Questions two and three are the same as three and four from the Gas Detector and Statistics questionnaire.
13 students agreed, or strongly agreed, that they enjoyed working with the Raspberry Pi while two students were neutral and one responded with 'not applicable'.
13 students agreed, or strongly agreed, to enjoying using the Arduino.
One student was neutral and three responded with 'not applicable.
These results are shown in Tab.\ref{tab:t12q23}.


\begin{table}[htpb]
  %%  \resizebox{\textwidth}{!}{%
  \scriptsize
  \begin{center}
    \caption{\scriptsize Student responses to the questions two and three of the Detector Principles Questionnaire.}
    \label{tab:t12q23}
    \begin{itemize}
    \item 2) Did you enjoy using the Raspberry Pi as the data collection medium?
    \item 3) Did you enjoy working with the Arduino Project?
    \end{itemize}
    

    \begin{tabular}{|c | c | c | c | c | c | c|}
      \hline
      Subject & SA & A & N & D & SD & N/A\\
      \hline
      Raspberry Pi & 3 & 10 & 0 & 0 & 0 & 1\\
      \hline
      Arduino & 3 & 10 & 0 & 0 & 0 & 3\\
      \hline
    \end{tabular}
  \end{center}%%}
\end{table}


Question four asks the students to respond to how interesting they found the new technology.
These results can be found in Tab.\ref{tab:t12q4}.
11 students agreed, or strongly agreed, that the new combination of the Raspberry Pi, Arduino, and GM counter made the experiment more interesting.
Two students disagreed, while three students responded neutrally, and one responded with 'not applicable'.

\begin{table}[htpb]
  %%  \resizebox{\textwidth}{!}{%
  \scriptsize
  \begin{center}
    \caption{\scriptsize Student responses to the question four of the Detector Principles Questionnaire.}
    \label{tab:t12q4}
    \begin{itemize}
    \item  4) Did the combination of new technology for RaspberryPi/Geiger/Arduino make the experiment more interesting?
    \end{itemize}
    

    \begin{tabular}{|c | c | c | c | c | c | c|}
      \hline
      Subject & SA & A & N & D & SD & N/A\\
      \hline
      Raspberry Pi/Geiger/Arduino & 4 & 7 & 3 & 2 & 0 & 1\\
      \hline
    \end{tabular}
  \end{center}%%}
\end{table}


Questions five and six ask the students to evaluate how much the felt that they learned by working with the Raspberry Pi and Arduino, respectively.
Five students responded that they learned some new knowledge about working with the Raspberry Pi.
11 students responded that the learned very little or nothing about the Raspberry Pi, while 1 responded with 'not applicable'.
For the Arduino, 7 students responded that they learned some new knowledge about using the Arduino.
Nine students said that they learned nothing, or very little, with one responding 'not applicable'.
These results are shown in Tab.\ref{tab:t12q56}.

\begin{table}[htpb]
  %%  \resizebox{\textwidth}{!}{%
  \scriptsize
  \begin{center}
    \caption{\scriptsize Student responses to the questions five and six of the Gas Detector and Statistics Questionnaire.}
    \label{tab:t12q56}
    \begin{itemize}
    \item 5) How much did you learn about working with Raspberry Pi?
    \item 6) How much did you learn about working with Arduino?
      
    \end{itemize}
    

    \begin{tabular}{|c | c | c | c | c | c | c|}
      \hline
      Subject & quite alot & some & very little & N/A\\
      \hline
      Raspberry Pi & 0 & 5 & 9 & 2 & 1 \\
      \hline
      Arduino & 0 & 7 & 3 & 6 & 1 \\
      \hline
    \end{tabular}
  \end{center}%%}
\end{table}


As the Arduino compatible gas GM is new and can be used for multiple purposes, question seven asks the students if they would be interested in gaining more hands-on experience with the counter in a future lab experiment.
These results are displayed in Tab.\ref{tab:t12q7}.
11 responded to this with 'quite alot' or 'some'.
Six responded no, or very little, with one answering 'not applicable'.

\begin{table}[htpb]
  %%  \resizebox{\textwidth}{!}{%
  \scriptsize
  \begin{center}
    \caption{\scriptsize Student responses to the question seven of the Detector Principles Questionnaire.}
    \label{tab:t12q7}
    \begin{itemize}
    \item  7) Would you enjoy working with the GM counter in a more hands-on way?
    \end{itemize}
    

    \begin{tabular}{|c | c | c | c | c | c|}
      \hline
      Subject & quite alot & some & very little & none & N/A\\
      \hline
      Geiger & 4 & 6 & 4 & 2 & 1 \\
      \hline
    \end{tabular}
  \end{center}%%}
\end{table}


The students were also asked for free responses regarding what they liked, didn't like, and suggestions for improvement regarding the experiment.
Nine students responded that they found the setup easy to use.
Two students responded that they found the Python code easy to modify if they needed to make small changes.
Students also commented that the experiment was fast, automated, interesting (no further elaboration on specific interesting items), the setup was inexpensive, the data was in a digital format.
One student particularly enjoyed that the setup allowed him to better understand how the data is collected and stored.

Question 9 asked students what they didn't like about the experiment.
Only three responses were submitted.
Two students responded that the data needed to be recorded by hand.
This was due to a technical problem with the Python compiler and was corrected in time for the next set of students.
The third student responded that there was no internet available to the Raspberry Pi.
In this case, a temporary (longer than one day but less than two days) Network issue occurred.

Question 10 gave the students an opportunity to suggest changes for the instructors to make moving forward.
There were seven responses to this item.
Two students suggested a way to control the angular measurement using a stepper motor and the Arduino.
This was an interesting suggestion from the students and one that we will follow up on.

Two students commented that they would like it made possible to have a Python script to completely automate the data collection.
It would be interesting to add a part to accomodate this in the lab manual to give the students some experience with Python programs that call other Python programs.
If time is short, the instructor could provide a pre-made program that does this.

One student suggested adding some sort of sound producing mechanism that notifies the user when the data run has finished.
This is another interesting idea that could easily be implemented.
A piezo buzzer that is controlled by the Arduino based on input from a Python program is one way to accommodate this.
This would be added to the manual such that students can implement this.
Another student made the comment that more information on how the experiment works would be nice.
Another student responded that they would like to work more independently and write all of their own code to perform the experiment.


We received great feedback from the students on how we could improve this portion of the experiment.
Many students did not enter the experiment with much experience using micro-controllers.
A majority of the students that performed this lab said that they learned something and enjoyed using the micro-controllers and the GM counter.
However there were not very many enthusiastic responses to this portion of the experiment.
This could be due to the mostly cookie-cutter, guided presentation of the material.
Most of the code is pre-made and setup for students to use and only requires minor changes.
This caused the students to say that the experiment is easy to use and some responses that the students would like to be more responsible in developing their own data collection programs.
The students also made some great suggestions for improving the hardware that would simplify or improve the data collection process for this experiment.

\subsection{Stern-Gerlach}
This experiment evolved on a much larger level during the course of the semester than the other experiments detailed in this article.
One particular issue that was encountered, which is made clear from the student feedback, was that students who used exclusively apple products were not able to access a free serial Bluetooth application (app) for their iPhone.
There were several options that required spending money to purchase an app that would be compatible, however the instructors felt that it would be unfair to require some students to spend money on an app but not others.
This led the instructors to try to an alternative to depending on the Android serial app.

The attempted solution to this problem was to remove the HC-05 bluetooth modules and simply replace the communication medium with a wired USB connection from the Arduinos to the student laptops.
This method worked better, but some problems were still encountered in this mode.
For several students, their laptops did not have USB ports and were not able to connect to the Arduinos via the USB cable.
They also were unable to download an app to collect data.
These particular students were forced to revert back to the analog data collection method which we were trying to replace.
Also, it was difficult for the students to write code to simultaneously read data from more than a single Arduino at a time.
The instructors were unable to provide a solution to this in time to benefit students prior to the end of the semester.
The students were still able to read all of the values one at a time, however they had to accommodate the inconvenience of switching between USB connections.

To address these issues moving forward a Raspberry Pi with appropriate accommodations for the students to be able to record data properly will be used.
Attempts will also be made to better understand how to comply with the large diversity in smart-phone and laptop technology so that we can continue to engage students in these ways in future.

A questionnaire along similar lines to the Gas Detector and Statistics as well as the Detector Principles labs were administered to collect student feedback.
As a reference, 22 students responded to the questionnaire, though not all students responded to each question.
The first question attempted to get an idea of how much experience students had with Arduino micro-controllers and Bluetooth equipment prior to the experiment.
The results are shown in Tab.\ref{tab:t14q1}.
19 students responded that they had no prior experience with either new type of technology we were implementing in this experiment.
Only a single student responded that they had 'some' prior experience while no students responded that they had 'significant' or 'expert' experience.
These responses have the same designation as described for the first question of the Gas Detector and Statistics questionnaire.

\begin{table}[htpb]
  %%  \resizebox{\textwidth}{!}{%
  \scriptsize
  \begin{center}
    \caption{\scriptsize Student responses to the first question of the Stern-Gerlach Questionnaire.}
    \label{tab:t14q1}
    \begin{itemize}
    \item 1) How much experience with Arduino and/or Bluetooth equipment did you have prior to this experiment?
    \end{itemize}
    

    \begin{tabular}{|c | c | c | c | c |}
      \hline
      Subject & none & some ($<$ 3 yr) & significant  ($>$ 3 yr) & expert\\
      \hline
      Arduino+Bluetoth & 19 & 1 & 0 & 0\\
      \hline
    \end{tabular}
  \end{center}%%}
\end{table}

Questions two and three asked the students to gauge how much they enjoyed working with the smart-phone data collection and the Arduino, respectively.
7 students agreed, or strongly agreed, that they enjoyed working with the smart-phone data collection.
4 students disagreed, or strongly disagreed, and did not enjoy working with the new technology.
6 students responded neutrally, while 4 students responded with 'not applicable' and were students that utilized the USB readout.
12 students agreed, or strongly agreed, that they enjoyed working with the Arduino equipment.
No students responded that they did not enjoy working with the Arduino.
9 students responded neutrally, while 1 student responded with 'not applicable'.
These results are displayed in Tab.\ref{tab:t14q23}.

\begin{table}[htpb]
  %%  \resizebox{\textwidth}{!}{%
  \scriptsize
  \begin{center}
    \caption{\scriptsize Student responses to the questions two and three of the Stern-Gerlach Questionnaire.}
    \label{tab:t14q23}
    \begin{itemize}
    \item 2) Did you enjoy using the smart-phone as the data collection medium?
    \item 3) Did you enjoy working with the Arduino Project?
    \end{itemize}
    

    \begin{tabular}{|c | c | c | c | c | c | c|}
      \hline
      Subject & SA & A & N & D & SD & N/A\\
      \hline
      Smart-phone & 2 & 5 & 6 & 3 & 1 & 4\\
      \hline
      Arduino & 1 & 11 & 9 & 0 & 0 & 1\\
      \hline
    \end{tabular}
  \end{center}%%}
\end{table}

Question four asked the students to evaluate if they felt that the combination of the new technology incorporated made the experiment more interesting.
10 students answered that they agreed, or strongly agreed, that the inclusion of the new technology made the experiment more interesting.
Six students responded that they disagreed, or strongly disagreed.
Six students responded neutrally.
These results can be seen in Tab.\ref{tab:t14q4}.

\begin{table}[htpb]
  %%  \resizebox{\textwidth}{!}{%
  \scriptsize
  \begin{center}
    \caption{\scriptsize Student responses to the question four of the Stern-Gerlach Questionnaire.}
    \label{tab:t14q4}
    \begin{itemize}
    \item  4) Did the combination of new technology for smart-phone/Bluetooth/Arduino make the experiment more interesting?
    \end{itemize}
    

    \begin{tabular}{|c | c | c | c | c | c | c|}
      \hline
      Subject & SA & A & N & D & SD & N/A\\
      \hline
      Smart phone/Bluetooth/Arduino & 2 & 8 & 6 & 5 & 1 & 0\\
      \hline
    \end{tabular}
  \end{center}%%}
\end{table}

Questions five, six, and seven gave the students a chance to give more detailed feedback about what they liked, disliked, and would improve about the experiment, respectively.
Question five asked the students for feedback about what they liked about the experiment.
The responses fit into six generic categories, however two responses address more the potential that the experiment has than the experiment's current state.
The overwhelmingly largest response group is that students liked that the experiment worked well, or felt that it was easy to use.
Four students responded that they liked that liked working with new, or more modern technology.
Two students particularly liked being able to use their own devices.
These students did not specify whether they used their smart-phones or laptops.
One student responded that they appreciated how accessible and well documented the experiment was.
Another stated that they enjoyed that the experiment was platform independent.
It is unclear what this student is referring to, whether it is that there exist apps for both iPhone and Android that work, or if either a smart-phone or laptop could be used to perform the experiment.

Two students appeared to comment on the potential upside that the experiment has in being an interesting lab for students.
One student stated that ``(The experiment has) alot of potential when used properly''.
Another student responded that ``If (the experiment) had worked right, it would've been cool''.
Both of these responses speak to the fact that they experienced some of the difficulties described earlier in this section regarding this experiment.
They also speak to the idea that these students would have been impressed by the new technology had it worked properly.
Some of the ideas on how to improve the experiment were already discussed earlier, but it suggests the benefit to working towards a properly functioning setup is worth making.

Question six asked students to respond to what they didn't like about the experiment.
The complaints collected from the students are mostly expected based on the previous discussion in this section.
The response fit into six main categories.
The first category is that the technology used is not sufficient to allow students to complete some, or in certain cases all, of the experiment without asking for additional resources from the instructor.
10 students registered complaints within this category.
Student complained about not being able to keep their laptop charged or not having a USB port on their laptop.
Other students complained that there was no free app for iPhone.

The second largest group of complaints fit into the realm of fluctuating values in the readout.
Seven students made complaints along these lines.
This is primarily due to the precision of the ADS1115 due to the gain.
In future iterations, it will be possible to take a time average of perhaps 10 or 20 readings that are updated less frequently than once per second.
Most of the reading do not need to be read out that often.

The third group of complaints concerned the fact that there was no automated data logging system available for this experiment.
Seven students made this type of complaint.
This is one place that the instructors can certainly make an improvement.
With the inclusion of the Raspberry Pi discussed earlier, this is easily done with the inclusion of a simple Python program.

Six students registered complaints in the next most popular category of having difficulty reading multiple serial connections.
This is an understandable complaint.
This can easily be solved using Python programming to multiplex the many Serial connections.

One student commented that the setup for the experiment did not look good.
The setup had many breadboards and Arduinos.
Moving forward it will be possible to put the different Arduino circuits in their own housings.
Ultimately, the instructors would like to combine the many different readings to be read by as few Arduinos as possible.
Afterall, each ADS1115 can read up to four different readings (provided that the maximum current and voltage limits are respected).
It is even possible to utilize up to four ADS1115s with the same Arduino when utilizing the appropriate addressing for each of the ADS1115s.

Another student made the comment that they would have liked to write their own Python code and be responsible for developing their own portion of the experiment.
There are certainly places within the experiment that could accommodate this.

In the seventh, and final question, students were asked to provide recommendations for improving the experiment.
The responses fit into six general categories.
The category that the most students responded with was  along the lines of recording or storing data more easily.
Eight students responded with suggestions to do this primarily with Raspberry Pi or Python.

Five students recommended rendering the data on a screen or with a GUI.
Many of them suggested doing this using the Raspeberry Pi.
Three students made recommendations along a third category of simplifying the experiment setup.
As stated earlier, this can likely be done by combining some readings with the same Arduino+ADS1115.

Another category that saw three student responses was that of improving the smart-phone comparability of this lab.
Two students recommended making the experiment Python compatible.
And one final student recommendation was to let students attempt to write code prior to attending the lab to get more hands-on learning/experience out of the experiment.

Many of the items have been addressed in previous parts of this section, but we would like to summarize them here.
There were several improvements to the experiment that were made 'on the fly' during the semester in response to some technological issues that were encountered.
The primary problem was that any additions for smart-phone need to be compatible with both Android and IOS platforms.
Unfortunately, the oversight of having only Android hindered the success of the additions to this experiment.
From this one could also have a Raspberry Pi, or some other computer based option.
Secondly, when switching over to the USB cable interface for laptops, some students had issues keeping their devices charged or there was an insufficient number of USB ports on their device.
Firstly, reducing the number of Arduinos needed for the experiment, by combining the measurements to be collected by the same one, or few, Arduinos would simplify the experiment setup and also help the second issue.
Secondly, using a Raspberry Pi provided by the course would also relieve the problem posed by students keeping requiring the need for keeping their devices charged.
Lastly, the instructors would like to provide a way of automating the data logging, which should be straight forward with e.g. Raspberry Pi.

\section{Summary}
In summary, three experiments, or portions thereof, were modified using microcontroller technology coupled with either Python programming or provisions for reading data via smartphones.
In the Gas Detector and Statistics experiment, an Arduino compatible PIN GM was used in conjunction with an Arduino Nano and Raspberry Pi.
Students were expected to write their own Arduino data acquisition code and Python data storage code.
Students, overall, generally enjoyed this experiment and felt that the inclusion of the new technology helped to make the experiment more interesting.
Students particularly felt that they got extra experience, and one could infer, increased self sufficiency working with this experiment.
Students felt that there was more transparency in how the data was collected, stored, and analyzed compared to a standard experiment that operates as more of a black box.
In short, there are only minor improvements that we would make to this lab.
Namely, in finding better ways to present material in the lab manual.
Another improvement would be to, possibly, allow students to start writing code prior to the experiment to allow sufficient time for debugging.

In the Mulitple Scattering portion of the Detector Principles experiment, a single, more traditional, gas GM was replaced by a series of four Arduino compatible gas GMs controlled by an Arduino Uno with data stored on a Raspberry Pi.
In this experiment, the goal of modernization was to simply reduce the data collection time by increasing the number of data points collected simulataneously using modern microcontroller technology.
Another goal was to find a better way to store the data using the Raspberry Pi.
Students had a little less freedom to develop their own programs in this experiment compared to that of the Gas Detector and Statistics experiment.
This was intentional as the Multiple Scattering experiment comprises one third of the amount of experiments in the Detector Principles lab, which is already typically running short on allotted time.
Students, overall, appreciated the efforts of modernizing this lab.
They liked that there were new technology that they had the opportunity to try out.
The students responded that they would like to have further automatization of the data collection as well as controlling the angular position of the detectors using the Arduino.
Controlling the angle would be straight-forward using a stepper motor, or something similar.
The automation of the data collection could be done using Python scripting, or something similar by the students.
In short, there are some minor improvements that can be done to improve the experiment for the students.

In the Stern-Gerlach, the third and final experiment which we added Arduino technology to, we simply replaced all of the analog readouts with Bluetooth serial readouts.
Thus, students would be able to use their own smartphones with Android serial terminal applications to read the data.
Several issues were found with this mode of data acquisition.
The most important of these being that the Apple iStore does not support a free application that is compatible with the experiment for students with Apple devices.
Secondarily, some students do not have the ability to keep their devices charged.
Thirdly, students could only listen to a single bluetooth+Arduino combination.
Lastly, there was not a good way of storing the data in a way that students could utilize it.

After several attempts using the smartphone readout, it was abandoned in favor of USB to laptop readout.
This worked better, solving the issue with the iOS app and storing data.
However, some students still had difficulty because they often had an insufficient number of USB ports for the experiment.
In addition, even if the students had sufficient USB ports to accomadate the large number of Arduinos used in the experiment, they were still unable to read data from more than one Arduino at the same time.

Students, overall, generally liked the type of technology that was being introduced into this experiment.
However, students would have appreciated the experiment changes that were made if they worked with little to no difficulty.
Additionally, students voiced that they would like more freedom to make changes to the coding and data acquisition process.
Most of this can be accomplished easily by introducing a Raspberry Pi and reducing the number of Arduinos needed to perform all of the measurements required for this laboratory.

All of these changes that we made to the laboratory experiments were made in an attempt to modernize them and engage students in a more positive way.
Based on the feedback from the students, we would say that we were, overall, successful in doing so.
We also understand that there are certain cases where some, not insignificant, improvement is necessary prior to the next iteration of the laboratory course.
However, thanks to the student feedback we know where most of the important focus points for the discussed experiments are.
Also, thanks to the overwhelmingly positive feedback, we also know that students enjoy working with this type of equipment and enjoy being given more freedom to develop theor own solutions to collecting and analyzing the data in most experiments.
Thus, this gives a starting point for not only improving other existing experiments within the course, but also in designing new experiments moving forward.



\end{document}
